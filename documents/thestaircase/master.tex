\documentclass{article}
\usepackage{hyperref}
\usepackage{tabularx}

\title{The Staircase}
\date{October 09 2012}
\author{JaKaTaKSc2}

\begin{document}
\maketitle\newpage
\tableofcontents\newpage
\section{Alternate Improvement Method}
\paragraph{}
TheStaircase is an alternative improvement method that focuses on creating a
fun, and motivating experience for the player. It utilizes unit constrictions to
prevent the player from being overwhelmed by the complexity of the game, and to
encourage creativity and exploration. It uses measurable benchmarks for the
absolute core elements of Starcraft 2: Gathering and spending resources
efficiently. These benchmarks are recorded using GGTracker, allowing the player
to devote all of their attention to playing the game.


\section{What to expect}
\paragraph{}
The Stuff League is the league in which, a player that consistently has more
“stuff” than their opponent will win more than 50% of their games. Currently The
Stuff League encompasses the Bronze, Silver, and Gold Leagues. That’s right, all
is required to obtain a promotion to the Gold League is the ability to make more
stuff than your opponent (more commonly referred to as macro). This is the point
where most readers will experience a healthy amount of doubt and skepticism. Not
to worry, the ladder is waiting to challenge this theory: prepare to be 
surprised. Professional players have proven this again and again, some making
it as far as Diamond League with mass drone! However this is easier said than
done, and the classic advice “just macro better” is still as annoying and
unhelpful as ever.

\paragraph{}
TheStaircase is designed to help guide you through the basics of Starcraft 2 by
playing games. There is no out of game practice, study, or memorization needed.
With this method you can jump right on the ladder and start winning games now,
even if you’ve never played Starcraft before. You will start with the most basic
units and explore their strengths and weaknesses by playing games and
experimenting. This will be guided by the Spending and Saturation benchmarks.
Focus on hitting your benchmarks and work your skills up from bronze to
grandmaster. Once you’re a grandmaster mineral-only macro-er, you’re ready to
move on to the next step. After completing TheStaircase you will have a solid
foundation of mechanics that will make learning strategies, build orders, and
keeping up with the meta-game of Starcraft 2 easier and more fun. Let’s begin.


\section{Getting started}
\paragraph{}
The goal is not to win now. The Goail is to improve now. Do not try to win now.
Try to improve now, and you will win later.

\paragraph{}
TheStaircase is a training system designed to help you reach your league goal 
in the most efficient way possible.  Choose the league you want to reach below 
and follow the corresponding benchmarks to achieve this.  Keep in mind, that 
this should be your long term goal and is often times not just one league 
above your current league.

\paragraph{TheStaircase (How to)}
\begin{enumerate}
    \item Choose a Race.
    \item Choose a League Goal.
    \item Start at Step 0 if Zerg. Start at Step 1 if Terran or Protoss.
    \item Follow the constrictions for each step.
    \item To pass a game, you must meet or exceed all benchmarks in your league
        goal.
    \item When you pass 4 out of 5 games you move onto the next step.
\end{enumerate}

\paragraph{The Staircase}
\begin{description}
    \item[Step 0] Play ``Creep or Die" in the arcade and use ``TheStaircase"
        mode against no opponent. 
    \item[Step 1] Play a melee game* of Starcraft 2 (Mine only Minerals. (no
        gas) Attack Move on the minimap only. (do not look at battles)
    \item[Step 2] Mine only minerals (no gas) and Control your units to the best
        of your ability.
    \item[Step 3] Mine Gas and Minerals and Control your units to the best of
        your ability. Given Units only.
    \item[Step 4] Choose 1 unit from the Battle Units List. Aside from the Given
        Units, this is the only unit you may build.
    \item[Step 5] Choose 2 units from the Battle Units List. Aside from the
        Given Units, these are the only units you may build.
    \item[Step 6] Choose 3 units from the Battle Units List. Aside from the
        Given Units, these are the only units you may build.
    \item[Step 7] Repeat Step 5 until you are satisfied with your progress and
        are ready to move on to build orders and strategy.
\end{description}

\paragraph{The Rules}
\begin{enumerate}
    \item NO clicking in the command card  (bottom right portion of the game
        screen).
    \item NO more than 2 queued units per production facility.
    \item NO more than b+2 supply structures in construction at a time, where 
        b~=~the~number of mineral saturated bases.
        
        Mineral Saturation occurs when there are 16 workers mining minerals.
\end{enumerate}

\begin{table}[h]
    \caption{Given and Battle units}
    \begin{tabularx}{\linewidth}{|X|X|X|}
         \hline
         \multicolumn{3}{|c|}{\textbf{Units}}\\
         \hline
         \textbf{Protoss} & \textbf{Terran} & \textbf{Zerg} \\
         \hline
         \textbf{Given Units} & \textbf{Given Units} & \textbf{Given Units} \\
         \hline
         Probe & SCV & Drone\\
         \hline
         Zealot & Marine & Overlord\\
         \hline
         Mothership Core / Mothership & Hellion & Zergling\\
         \hline
         Observer & Hellbat & Queen\\
         \hline
         Warp Prism &  & Overseer\\
         \hline
         \textbf{Battle Units} & \textbf{Battle Units} & \textbf{Battle Units}\\
         \hline
         Stalker & Marauder & Baneling\\
         \hline
         Sentry & Reaper & Roach\\
         \hline
         Immortal & Thor & Hydralisk\\
         \hline
         Colossus & Siege Tank & Ultralisk\\
         \hline
         Dark Templar / Archon & Ghost & Mutalisk\\
         \hline
         High Templar / Archon & Viking & Swarm Host\\
         \hline
         Void Ray  & Battlecruiser & Brood Lord\\
         \hline
         Oracle & Banshee & Infestor\\
         \hline
         Tempest & Raven & Viper\\
         \hline
         Carrier & Widow Mine & Corruptor\\
         \hline
         Phoenix & Medivac & \\
         \hline
    \end{tabularx}
\end{table}

\paragraph{Benchmark Key}
\begin{description}
    \item[Spending Skill] An equation that measures the skill of keeping your
        money low relative to your income.
    \item[Saturation Speed] The time it takes from base completion to 16 workers
        on minerals
    \item[Benchmark Calculation] GGTracker.com
    \item[Consistency Requirement] Must hit both Benchmarks 4 out of 5 games to
        PASS
\end{description}

\paragraph{Guidelines/Notes}
\begin{itemize}
    \item Play aggressively; avoid static defenses
    \item Experiment with the new unit/units on your current Step as much as
        possible!
    \item Focus as much as possible on your goals, and not on winning the game.
    \item Avoid arrow keys or moving the mouse to the edge of the screen to
        move.
    \item Click on the minimap and use location keys to move screen
    \item Use Hotkeys, Control Groups, and Camera Location keys as much as
        possible.
    \item Try to get 48 workers on Step 1 (3 base saturation)
    \item Try to get 66 workers on all other Steps (3 base saturation)
    
        NOTE: only take the gas you need when you need it!!!
    \item Feel free to rearrange the order of units once you get past Step 2.
    \item Have as much fun as possible at all times  :D
    \item Try to keep at least one unit or group of units active on the map
        after your first scout, this will help train multitasking.
    \item For Step 2 and onward, try to look at your base only when necessary
\end{itemize}

\paragraph{Benchmark}
\begin{itemize}
    \item Create an account on GGTracker.com
    \item Click on TheStaircase view in the drop down menu under your name.
    \item Choose a League Goal: The league you want to be in, long term.
    \item Choose a Race: Zergs go to Step 0, Protoss and Terran go to Step 1
    \item To pass a game you must meet or exceed all benchmarks in your League 
        Goal.
    \item When you pass 4 out of your last 5 games Move on to the next Step.
\end{itemize}

\paragraph{}
The staircase YouTube tutorial video:
\url{https://www.youtube.com/watch?v=33NbuvSq8Kg}


\section{The benchmarks}
\paragraph{}
The foundation of Starcraft 2 is the gathering and spending of resources.
Without these two elements Starcraft 2 wouldn’t be what it is. The metrics we
have developed regarding these 2 aspects of the game are called Spending Skill
and Saturation Speed. 

\paragraph{}
Both metrics are measured by league. This means that each game will be compared
to the average game of each league and given a corresponding badge. So if you
spent your resources as well or better than the average gold league player, but
not as well as the average platinum league player you would receive a gold badge
in the Spending Skill column. The same can be said for Saturation Skill.

\paragraph{}
Both metrics are measured from the 8 minute mark in a game until the 30 minute
mark (in-game time). If a game exceeds 30 minutes, a snapshot of the metrics is
taken from that point in the game and those metrics appear on the score screen
of ggtracker. This helps to avoid super early game cheeses, in which the player
did not get adequate time to demonstrate their abilities; as well as super late
game scenarios in which the metrics lose their consistency. 

\paragraph{Spending Skill}
Spending Skill, to put it briefly, measures how fast you spend your resources in
relation to how fast you are gathering them. The specifics are best explained by
GGTracker: \url{http://ggtracker.com/spending_skill}

\paragraph{}
Spending Skill is an improvement on an earlier established metric called
Spending Quotient, and you can still see your SQ when you mouse over the league
badge in the Spending Skill column. The difference is that it takes game length
into account as well as a larger pool of data to assign leagues to particular
SQs. These metrics have been at the core of TheStaircase since its inception and
continue to prove to be essential to improvement in Starcraft 2.

\paragraph{Saturation Speed}
Saturation Speed is the time it takes for a player to fully saturate the mineral
patches of a base. This measurement is taken from the time the Nexus, Hatchery,
or Command Center is complete and in mining position to the time 16 workers are
mining minerals. Again, GGTracker explains it best:
\url{http://ggtracker.com/saturation_speed}

\paragraph{}
Saturation Speed completes the economic benchmark duo, ensuring that players
aren’t only spending their resources, but that they have resources to spend in
the first place. Before this metric, it was possible for a player to have a much
higher Spending Skill than their opponent, but a smaller army and economy.


\section{Why The Staircase works}
\paragraph{The LAN Experiment}
Starcraft 2 had just come out and I had planned a big LAN party with many of my
friends. Some of them had never played Starcraft at all before, but were very
excited about the new game and wanted to try it out. One of my friends, we’ll
call him Yukon Cornelius, had never played Starcraft before and came early so
that I could show him the ropes a bit before everything got started. I found a
safe build for him to learn and wrote it down for reference and tried to explain
as much as I could about the game: Unit compositions, counters, strategies, and
all the things that I learned to play the game. He was eager to learn and began
working on his build order. When the rest of my friends came and we started
playing games and setting up brackets, I checked in on Yukon to see how he was
doing. He had been playing for about 9 hours at that point and he was getting
very frustrated and not showing very much improvement. Many of my friends
chalked this up to the idea that``Starcraft is a really hard game", but I wasn’t
satisfied with stopping there. 

\paragraph{}
After watching him play, I came up with an experiment. I told him to only
focus on 3 things: Keep your money low, build pylons ahead of time, and build
only zealots and probes. The result was mind bending. He wasn’t only beating
some of my other friends who had just recently picked up the game; he was
beating me; he was beating my friends that had been playing in the beta, with
our build order notebooks, counters, and strategies; and he was having fun
doing it. I had many LAN parties to follow, and repeated this experiment; the
results were consistent. The players I taught with this approach had more fun,
improved faster and won. The players I taught with the strategic approach got
frustrated, overwhelmed, and lost.

\paragraph{}
\url{https://www.youtube.com/watch?v=u6XAPnuFjJc}

\paragraph{}
I would definitely describe playing Starcraft as requiring more than
``rudimentary cognitive skill" and therefore the motivators of Autonomy,
Mastery, and Purpose, apply here. Mastery is clearly a motivator for playing
Starcraft. Purpose is somewhat distorted as the mainstream opinion of video
games as a whole has been that they are a waste of time, that is changing
slowly, but Starcraft has, i think, a particular role to play in this problem.
It is an incredibly beneficial exercise and challenge. More and more studies
are coming out about the benefits of playing Starcraft both mentally and
physically. In the future I will be doing more research on this and collecting
all of these things in one form or another, but until then there's one more
motivator to address: Autonomy.

\paragraph{}
Now, if you've heard about this method before, you may be saying to yourself,
``Limiting which units you can make does not promote Autonomy!" That is correct
in a way, but let's look at the picture as a whole. Given complete Autonomy,
every player would have to learn for him or herself all of the basic and
unchanging things about Starcraft that we already know. On the other hand,
removing all autonomy and saying that there is one and only one thing you must
to do learn this game is very de-motivational. The goal of this system is to
give enough guidance so that the player has a purpose and direction, but not
so much so that the player isn't being encouraged to exercise their creativity
or to have fun through experimentation.

\paragraph{}
There is one more parallel I want to draw between Dan Pink's talk on
motivation, and Starcraft 2. He talks about how money is a motivator in jobs,
but that the best way to use money as a motivator is to take it off the table
so that the employee doesn't have to worry about money, and can put all of
their focus on the task at hand. This, to me, is like winning and losing in
Starcraft. If we can take the anxiety of winning and losing out of the
equation, the player can better focus on the task at hand, improvement.

\paragraph{The Myth of the Bronze Player}
Its been more than a year now that I've focused on trying to understand ``The
Bronze Player" and why he/she has such a hard time moving up in the ranks.
From my experience, most higher level players respond to this question by
saying things like, ``They don't care", ``They don't want it bad enough",
``They're just having fun" and ``They don't want to improve". Before launching
TheJaKaTaK on May 1st, 2012, I spend around 6 months in the bronze league,
meeting bronze players, talking with them after matches, asking them if
they're frustrated, and what their plan is for improvement. Some said things
like, ``Whatever, I don't care, I'm just having fun, it's just a game, etc."
but when I talked to them long enough, and they had enough time to cool down
about their loss to a worker rush or an unexpected attack, they often took it
all back, admitted they were frustrated and asked for help. This did not
happen once, but many times over. I think it makes perfect sense that the
biggest thing keeping bronze players from promotion is not that they don't
care, or aren't trying, but that they are frustrated and lost. Many of them
were trying build orders and strategies, but couldn't find the motivation to
stick with them because it felt boring to them.

\paragraph{}
It is important to note that Starcraft, like any other skill, requires time,
focus, and consistency to improve. There were some bronze players I spoke with
that simply weren't playing enough to get promoted. This is another thing that
higher level players often times note, sometimes in the abrasive and
unhelpful, ``Just play more games" style. Right now, you only need a couple
months where you are averaging 1-2 focused games a day to get yourself into
the silver league. But as you climb higher, you will need to put in more time,
fortunately, as you get higher in the leagues, the play becomes more
challenging and fun (IMO) so putting in more time will likely be a scheduling
problem and not a motivational problem. When setting your league goal, make
sure it is in line with the time you have to play.

\paragraph{Psychology and Starcraft}
Games use a variety of psychological methods to get you to play and to keep
playing. The Skinner Box is one of the more well known methods. Starcraft 2
has a form of this in the achievement system, but the reward isn't directly
tied to improvement, which is what I want to do. I began looking for a better
understanding of human motivation and found this talk by Dan Pink particularly
interesting.


\section{F.A.Q.}
\begin{enumerate}
\item What about upgrades? Which ones can I get?

Any of them! On Step 0 and 1 you will not have gas for upgrades, so you won’t
be able to get them, but that isn’t because they are a constriction, it is
because you don’t have gas.

\item Is it okay to just do 1 or 2 steps and then go on to “normal” play?

Of course it is! TheStaircase is a tool to build your foundation, it’s great
for returning players to get their mechanics back in shape quickly, or to fill
in the gaps of macro and micro you’ve been missing. Each person should use
their own judgement as to how TheStaircase can best help them to improve.

\item Do I have to win to pass?

Absolutely not. Part of what TheStaircase trains is the disconnection from the
need to win. Winning is a poor benchmark for progress, many times players
improve more after a loss than a win, so why judge our progress by something
like winning?

\item Can I rush or cheese?

Absolutely! You have the freedom to choose any strat and the benchmarks will
still work! Get creative, try things that sound fun to you. That is the most
important part.

\item Can I play vs the AI?

You must play melee Sc2. This is the only restriction. (game type shows up in
the lobby). Play 4v4s, play 2v2s, vs friends, on the ladder, vs AI, whatever
you think is most fun!
\end{enumerate}


\end{document}
